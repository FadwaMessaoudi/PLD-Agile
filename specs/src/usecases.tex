\part{Cas d'utilisation}

\section{Charger un plan de ville}

\subsection{Préconditions}

\begin{itemize}
\item Le fichier existe
\item Le fichier est lisble (notamment vis-à-vis des permissions)
\item Le fichier contient du XML bien formé et conforme au schéma
\item Il n'y a pas d'intersections ou de tronçon orphelins
\end{itemize}

\subsection{Scénario principal}

\begin{enumerate}
\item Le système demande à l'utilisateur un URI vers le fichier XML
\item L'utilisateur saisit l'URI du fichier
\item Le système parse le fichier pour peupler un DOM
\item Le DOM est analysé par le système pour générer la représentation du plan en mémoire conforme au modèle du domaine
\item Le plan en mémoire est converti en représentation graphique affichée à l'écran
\end{enumerate}

\subsection{Scénarii alternatifs}

\begin{itemize}
\item[3a] Le fichier n'existe pas, n'est pas lisible ou n'est pas un fichier XML bien formé
\begin{enumerate}
\item Le système affiche un message d'erreur
\item Le système demande la saisie d'une autre URI de fichier
\end{enumerate}

\item[3b] Le fichier n'est pas conforme au schéma
\begin{enumerate}
\item Le système affiche un message d'erreur
\item Le système demande la saisie d'une autre URI de fichier
\end{enumerate}

\item[4a] Le plan contient un tronçon ou une intersection orphelin
\begin{enumerate}
\item Le système affiche un message d'erreur
\item Le système demande la saisie d'une autre URI de fichier
\end{enumerate}
\end{itemize}

\section{Charger une demande de livraison}

\subsection{Préconditions}

\begin{itemize}
\item Le fichier existe
\item Le fichier est lisble (notamment vis-à-vis des permissions)
\item Le fichier contient du XML bien formé et conforme au schéma
\item L'utilisateur a chargé un plan de ville au préalable
\item Toutes les intersections mentionnées dans le fichier XML de la demande de livraison existent dans le plan de la ville
\end{itemize}

\subsection{Scénario principal}

\begin{enumerate}
\item Le système demande à l'utilisateur un URI vers le fichier XML
\item L'utilisateur saisit l'URI du fichier
\item Le système parse le fichier pour peupler un DOM
\item Le système récupère l'adresse de l'entrepôt à partir du DOM ainsi que l'heure de départ de la tournée
\item Le DOM est analysé par le système pour générer la liste des livraisons conforme au modèle du domaine
\item Les emplacements de livraisons sont affichés sur le plan graphique
\end{enumerate}

\subsection{Scénarii alternatifs}

\begin{itemize}
\item[3a] Le fichier n'existe pas, n'est pas lisible ou n'est pas un fichier XML bien formé
\begin{enumerate}
\item Le système affiche un message d'erreur
\item Le système demande la saisie d'une autre URI de fichier
\end{enumerate}

\item[3b] Le fichier n'est pas conforme au schéma
\begin{enumerate}
\item Le système affiche un message d'erreur
\item Le système demande la saisie d'une autre URI de fichier
\end{enumerate}

\item[4a] L'adresse de l'entrepôt ne correspond à aucune intersection
\begin{enumerate}
\item Le système affiche un message d'erreur
\item Le système demande la saisie d'une autre URI de fichier
\end{enumerate}

\item[4b] L'heure de départ est mal formatée (pas au format (h)h:(m)m:(s)s)
\begin{enumerate}
\item Le système affiche un message d'erreur
\item Le système demande la saisie d'une autre URI de fichier
\end{enumerate}

\item[5a] L'adresse d'un des lieux de livraison ne correspond à aucune intersection
\begin{enumerate}
\item Le système affiche un message d'erreur
\item Le système demande la saisie d'une autre URI de fichier
\end{enumerate}

\item[5b] L'heure d'une des plages horaires de livraison est mal formatée (pas au format (h)h:(m)m:(s)s)
\begin{enumerate}
\item Le système affiche un message d'erreur
\item Le système demande la saisie d'une autre URI de fichier
\end{enumerate}
\end{itemize}

\section{Calculer une tournée}

\subsection{Préconditions}

\begin{itemize}
\item L'utilisateur a chargé un plan de ville
\item L'utilisateur a chargé une demande de livraisons
\end{itemize}

\subsection{Scénario principal}

\begin{enumerate}
\item L'utilisateur demande à calculer une tournée de livraison
\item Le système analyse les données chargées au préalable pour déterminer la tournée optimale
\item La tournée proposée est affichée sur le plan graphique (les tronçons sur le chemin de la tournée sont mis en surbrillance et une flèche indique le sens de criculation)
\end{enumerate}

%\subsection{Scénarii alternatifs}

%\begin{itemize}
%\item[2a] Le système ne trouve pas de tournée satisfaisant les plages horaires imposées
%\begin{enumerate}
%\item Le système affiche un message d'erreur
%\item Aucune tournée n'est affichée et le système revient dans l'état précédent
%\end{enumerate}
%\end{itemize}